
\documentclass[a4paper] {article}

\usepackage{listings}
%\lstset{language=MATLAB, basicstyle=\ttfamily}
\lstset{
language=MATLAB,
basicstyle=\small\ttfamily,
numbers=left,
numberstyle=\tiny,
frame=tb,
columns=fullflexible,
showstringspaces=false
}

\usepackage{graphicx}

\title{\vspace{-1cm} Problem 6}
\date{}
\author{}

\begin{document}
\maketitle
(a) To generate Brownian motion paths in $[0, 1]$ using the Karhunen-Loeve expansion, first
define the following function.
  
\begin{lstlisting}[caption=BrownianMotionKL.m]
function W_t = BrownianMotionKL(t, num_terms)
% Return a vector with the values of a BM path sampled 
% at the elements of
% t, using the Karhunen-Loeve expansion
%   t         - Row vector of sample points
%   num_terms - Number of terms in the Karhunen-Loeve expansion

W_t = zeros(1, length(t));
Z   = randn(1, num_terms);

% It is more convenient to start the sum from k = 1.
for k = 1:num_terms
	coeff = sqrt(2) * Z(k) / ((k - 0.5) * pi);
	term  = coeff   * sin((k - 0.5) * pi * t);
	W_t   = W_t + term;
end

end
\end{lstlisting}

and then execute the following script.

\begin{lstlisting}[caption=PlotBrownianMotionPaths.m]
figure; 
hold on;
title('Brownian motion paths in [0, 1], J = 5000');
xlabel('t')
ylabel('W(t)')

t = linspace(0, 1, 2000);

% plot 10 brownian motion paths
for k = 1:10
    y = BrownianMotionKL(t, 5000);
    plot(t, y);
end

plot([0 1], [0 0], 'color', 'black');

ylim([-2.5 2.5]);

hold off;
\end{lstlisting}

\begin{figure}[h]
\centering
\includegraphics[scale=0.5]{brownian_paths}
\caption{Sample script output.}
\end{figure}
\vspace{1cm}

(c) To plot $e_{M,J}(T)$, use the following code.

\begin{lstlisting}[caption=WJError.m]
function [e_t] = WJError(t, M, J)

e_t = zeros(1, length(t));
wjs = zeros(M, length(t));

for k = 1:M
    wjs(k, :) = BrownianMotionKL(t, J);
end

wjs         = wjs .^ 2;
column_sums = sum(wjs); 

e_t = abs( (1 / M) * column_sums - t);

end
\end{lstlisting}

\begin{lstlisting}[caption=PlotWJError.m]
J = floor(logspace(0, 3, 15)) + 1;
t = linspace(0, 1, 100);

figure;
hold on;

for ii = 1:length(J)
    y = WJError(t, 4000, J(ii));
    plot(t, y);
end

hold off;
\end{lstlisting}

This ouputs the following.

\begin{figure}[h]
\centering
\includegraphics[scale=0.5]{wjerror_paths}
\end{figure}

\end{document}
